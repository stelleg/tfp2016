\subsection{Results} \label{sec:res}

Figure~\ref{fig:res} gives the benchmark results.  In general, we are
outperformed by GHC, sometimes significantly, and we outperform UHC. We
spend the rest of the section analyzing these performance differences.

\begin{figure*}
\centering
\begin{tabularx}{\textwidth}{l | X | X | X | X | X}
& $\mathcal{CE}$ & GHC -O0 & UHC -O0 & GHC -O3 & UHC -O3 \\
\hline
\texttt{exp3 8} & 1.530 & 1.176 & 3.318 & 1.038 & 2.286 \\
\texttt{tak 16 8 0} & 0.366 & 0.146 & 1.510 & 0.006 & 1.416 \\
\texttt{primes 1500} & 0.256 & 0.272 & 1.518 & 0.230 & 1.532 \\
\texttt{queens 9} & 0.206 & 0.050 & 0.600 & 0.012 & 0.598 \\
\texttt{fib 35} & 2.234 & 0.872 & 10.000 & 0.110 & 8.342 \\
\texttt{digits-of-e1 1000} & 3.576 & 1.274 & 21.938 & 0.118 & 22.010 \\
\texttt{digits-of-e2 1000} & 0.404 & 0.792 & 3.430 & 0.372 & 3.278 \\
\texttt{fannkuch 8} & 0.560 & 0.084 & 2.184 & 0.048 & 2.196 \\
\end{tabularx}
\caption{Machine Literals Benchmark Results. Measurement is wall clock time,
units are seconds. Times averaged over 5 runs.}
\label{fig:res}
\end{figure*}

There are many optimizations built in to the abstract machine underlying GHC,
but three in particular are likely the cause of much of the performance disparity: 

\begin{itemize}
\item \textbf{Register allocation:} The $\mathcal{CE}$ machine has no register
allocator. In contrast, by passing arguments to functions in registers, GHC
avoids a lot of heap thrashing.
\item \textbf{Unpacked Literals:} Related to register allocation, this allows
GHC to keep machine literals without tags in registers for tight loops. In
contrast, the $\mathcal{CE}$ machine operates entirely on the stack, and has a
code pointer associated with every machine literal. 
\item \textbf{Y combinator:} For example, \texttt{tak} is largely a benchmark of
recursion.  Because recursion in the $\mathcal{CE}$ machine is implemented with
a Y combinator, it performs poorly. This should be alleviated with CFA-based
techniques, similar to those used in ~\cite{rozas1992taming}. 
\end{itemize}

Lack of register allocation is the primary drawback of the $\mathcal{CE}$
machine. Unlike the STG machine, which pulls the free variables into registers,
allowing tight loops with everything kept in registers, it is much less clear
how to effectively do register allocation in a fully shared environment setting.
Of course, we'd like to avoid traversing the shared environment more than
absolutely necessary. On this flip side, just as with constructing the delayed
computation, the \emph{loading} the computation is made lazier in a sense. That
is, we don't load values into registers until they are forced. Of course, this
can lead to memory thrashing. This trade-off between lazy register loading and 
memory thrashing is an interesting area for future work.

We attempt to isolate the effect of register allocation and unpacked machine
literals by replacing the use of machine integers with Church numerals for a
compatible subset of the evaluation programs. See Figure~\ref{fig:res-church}
for performance with this modification. Comparative performance is much
improved, with the $\mathcal{CE}$ machine occasionally even outperforming
optimized GHC.

\begin{figure*}
\centering
\begin{tabularx}{\textwidth}{l | X | X | X | X | X}
& $\mathcal{CE}$ & GHC -O0 & UHC -O0 & GHC -O3 & UHC -O3 \\
\hline
\texttt{tak 14 7 0} & 1.610 & 2.428 & 7.936 & 1.016 & 7.782 \\
\texttt{primes 32} & 0.846 & 1.494 & 4.778 & 0.666 & 5.290 \\
\texttt{queens 8} & 0.242 & 0.374 & 1.510 & 0.154 & 1.508 \\
\texttt{fib 23} & 0.626 & 0.940 & 5.026 & 0.468 & 5.336 \\
\texttt{digits-of-e2 6} & 0.138 & 1.478 & 5.056 & 0.670 & 5.534 \\
\texttt{fannkuch 7} & 0.142 & 0.124 & 0.796 & 0.040 & 0.808 \\
\end{tabularx}
\caption{Church Numeral Benchmark Results. Measurement is wall clock time, 
units are seconds. Times averaged over 5 runs.}
\label{fig:res-church}
\end{figure*}

We go even further, and remove the disparity due to the Y-combinator, by
running a simple example with church numerals. We use a simple church
numeral exponentiation, calculating $3^8 - 3^8$. We see that in this case, we
significantly outperform both GHC and UHC, as seen in
Figure~\ref{fig:res-church-exp} .

\begin{figure*}
\begin{tabularx}{\textwidth}{l | X | X | X | X | X}
& $\mathcal{CE}$ & GHC -O0 & UHC -O0 & GHC -O3 & UHC -O3 \\
\hline
\texttt{pow 3 8} & .598 & 3.766 & 4.302 & 3.988 & 3.980 \\
\end{tabularx}
\caption{Church Numeral Exponentiation Benchmark Results. Measurement is wall clock time, 
units are seconds. Times averaged over 5 runs}
\label{fig:res-church}
\end{figure*}

We expect that with the addition of the optimizations mentioned above, as well
as others, the $\mathcal{CE}$ machine will become significantly more
competitive. We discuss how some of these optimizations can be applied to the
$\mathcal{CE}$ machine in Section~\ref{sec:disc}.

\subsection{The Cost of the Cactus}

Recall that variable lookup is linear in the index of the variable, following
pointers until the index is zero. As one might guess, the lookup cost is very
large. For example, for the queens benchmark without any optimizations, variable
lookup took roughly $80-90\%$ of the runtime, as measured by profiling. Much of
that cost was for lookups of known combinators, however, so for the benchmarks
above we added the inlining mentioned in the previous section. Still, even with
this simple optimization, variable lookup takes roughly $50\%$ of the
runtime. There is some variation across benchmarks, but this is a rough
approximation for the average cost. We discuss how this cost could be addressed
in future work in Section~\ref{sec:disc}.


