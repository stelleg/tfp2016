\section{Introduction}

The key idea of the \emph{Cactus Environment Machine} is to use a shared
environment to minimize the cost of delayed computation, and thus implement
call-by-need efficiently.

To give an initial intuition for how it minimizes the cost of delayed
computation, consider the application $t \; t'$. In existing call by need
implementations, a closure with a flat environment will be constructed for $t'$.
This has a time and memory cost proportional to the number of free variables for
$t'$ \footnote{in some implementations, these will be lambda lifted to be formal
parameters, but the principle is the same}. In the case that $t'$ is unneeded,
this work is entirely wasted. In the spirit of lazy evaluation, the Cactus
Environment Machine attempts to minimize this potentially unnecessary work.  

The main contributions of the paper include:
\begin{itemize}
\item A calculus that formalizes a shared environment for call by need
evaluation using an explicitly shared environment (Section~\ref{sec:calc}).
\item An abstract machine (the $\mathcal{CE}$ machine) that implements the
calculus (Section~\ref{sec:mach}). 
\item A simple compiled implementation of the abstract machine and evaluation,
which shows performance comparable to existing implementations
(Sections~\ref{sec:impl} and~\ref{sec:eval}).
\end{itemize}

Section~\ref{sec:back} reviews relevant background material, and
Section~\ref{sec:env} discusses the current landscape of environment
representations, highlighting the opportunity for combining shared environments
with lazy evaluation.  We then provide some intuition for why this combination
is desirable, and formalize the connection between call by need evaluation and
shared environments in a calculus (Section~\ref{sec:calc}).
Section~\ref{sec:mach} uses the calculus to derive a novel abstract machine, the
$\mathcal{CE}$ machine, explains how $\mathcal{CE}$ uses the shared environment
in a natural way to implement lazy evaluation, and gives its formal semantics.
We then describe a straightforward implementation of $\mathcal{CE}$ in
Section~\ref{sec:impl}, extended with machine literals and primitive operations,
compiling directly to native code. We evaluate the implementation in
Section~\ref{sec:eval}, showing that it is capable of performing competitively,
despite lacking several common optimizations, and discuss the results. We
conclude by discussing related work, the limitations of our approach, and give
some ideas for future work in Section~\ref{sec:disc}, and conclude the paper in
Section~\ref{sec:conc}.


