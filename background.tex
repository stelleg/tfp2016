\section{Background and Motivation} \label{sec:back}

This section provides relevant background for the $\mathcal{\mathcal{C} \mskip -4mu \mathcal{E}}$ machine,
outlining lambda calculus, evaluation strategies, and Curien's calculus of
closures.

\subsection{Preliminaries}

We begin with the simple lambda calculus ~\cite{barendregt1984lambda}:  $$ t::= x
\; | \;  \lambda x.t \; | \;  t \; t $$ where $x$ is a variable, $\lambda x.t$
is an abstraction, and $t \; t$ is an application. We also use lambda calculus
with deBruijn indices, which replace variables with a natural number indexing
into the binding lambdas.  This calculus is given by the syntax: $$ t::= i \; |
\; \lambda t \; | \; t \; t $$ where $i \in \mathbb{N}$. In both cases, we use
the standard Barendregt syntax conventions, namely that applications are left
associative and the bodies of abstractions extend as far as possible to the
right ~\cite{barendregt1984lambda}.  A \emph{value} in lambda calculus refers to
an abstraction. We are concerned only with evaluation to weak head normal form
(WHNF), which terminates on an abstraction without entering its body.

In mechanical evaluation of expressions, it would be too inefficient to perform
explicit substitution. To solve this, the standard approach uses closures
~\cite{landin1964mechanical,curien1991abstract,jonesstg,biernacka2007concrete}.
Closures combine a term with an environment, which binds the free variables in
the term to closures. 

For a formal basis for closures, we use Curien's calculus of
closures~\cite{curien1991abstract}, given in Figure~\ref{fig:calcclos}.  It is a
formalization of closures with an environment represented as a list of closures,
indexed by deBruijn indices. We will occasionally modify this calculus by
replacing the deBruijn indices with variables for readability, in which case
variables are looked up in the environment instead of indexed, e.g., $t[x = c, y
= c'])$ ~\cite{barendregt1984lambda}. We also add superscript and subscript
markers to denote unique syntax elements, e.g., $t', t_1 \in \textnormal{Term}$. 

\begin{figure}
\textbf{Syntax}
\begin{align*}
\tag{Term} t &::= i \; | \; \lambda t \; | \; t \; t  \\
\tag{Variable} i &\in \mathbb{N}  \\
\tag{Closure} c &::= t [\rho] \\
\tag{Environment} \rho &::= \bullet \; | \; c \cdot \rho \\
\end{align*}
\textbf{Semantics}
\begin{align*}
\tag{LEval}\inference
{t_1[\rho] {\xrightarrow{* }}_L \lambda t_2[\rho'] }
{t_1 t_3[\rho] \rightarrow_L t_2[t_3[\rho] \cdot \rho'] } 
\end{align*}
\begin{align*}
\tag{LVar} i [c_0 \cdot c_1 \cdot ... c_i \cdot \rho] \rightarrow_L c_i
\end{align*}
\caption{Curien's call-by-name calculus of closures ~\cite{curien1991abstract}}
\label{fig:calcclos}
\end{figure}

\subsection{Evaluation Strategies} \label{sec:eval}

There are three standard evaluation strategies for lambda calculus:
call-by-value, call-by-need, and call-by-name.  Call-by-value evaluates every argument
to a value, whereas call-by-need and call-by-name only evaluate an argument if
it is needed.  If an argument is needed more than once, call-by-name re-computes
the value, where call-by-need memoizes the value, so it is computed at most once.
Thus, call-by-need attempts to embody the best of both worlds---never repeat
work (call-by-value), and never perform unnecessary work (call-by-name). These
are intuitively \emph{good} properties to have, and we illustrate the
correctness of such an intuition with the following example, modified from
~\cite{danvy2013synthetic}:

$$ \overbrace{c_m (c_m (\cdots(c_m}^{m} \; id \; \overbrace{id)\cdots) id)}^{m} \; true \; id
\; bottom $$ where $c_n = \lambda s.\lambda z.\overbrace{s \; (s \cdots (s}^{n}
\; z) \cdots) $, $true = \lambda t.\lambda f.t$, $id=\lambda x.x$, and $bottom =
(\lambda x.x \; x) \lambda x.x \; x$. Call-by-value never terminates,
call-by-name takes exponential time, and call-by-need takes only polynomial time
~\cite{danvy2013synthetic}. Of course, this is a contrived example, but it
illustrates desirable properties of call-by-need.

In practice, however there are significant issues with call-by-need evaluation.
We focus on the following: \emph{Delaying a computation is slower than
performing it immediately.} This issue is well known
\cite{johnsson1984efficient,jonesstg}, and has become part of the motivation
for \emph{strictness analysis}
\cite{mycroft1982abstract,wadler1987projections}, which transforms non-strict
evaluation to strict when possible.

\subsection{Existing Call-by-Need Machines}

Diehl et al. ~\cite{diehl2000abstract} review the call-by-need
literature in detail.  Here we summarize the most relevant points.

The best known machine for lazy evaluation is the Spineless Tagless
G-Machine (STG machine), which underlies the Glasgow Haskell Compiler (GHC). 
STG uses flat environments that can be allocated on the stack, the heap,
or some combination ~\cite{jonesstg}.  

Two other influential lazy evaluation machines relevant to the $\mathcal{\mathcal{C} \mskip -4mu \mathcal{E}}$
machine are the call-by-need Krivine machines
~\cite{lkm,krivine2007call,sestoft}, and the three-instruction machine (TIM)
~\cite{TIM}.  Krivine machines started as an approach to call-by-name
evaluation, and were later extended to call-by-need
~\cite{krivine2007call,sestoft,danvy2013synthetic,lkm}.  The $\mathcal{\mathcal{C} \mskip -4mu \mathcal{E}}$
modifies the lazy Krivine machine to capture the environment sharing given by
the cactus environment. The TIM is an implementation of call-by-need and
call-by-name ~\cite{TIM}.  It involves, as the name suggests, three machine
instructions, \texttt{TAKE}, \texttt{PUSH}, and \texttt{ENTER}. In
Section~\ref{sec:impl}, we follow Sestoft ~\cite{sestoft} and
re-appropriate these instructions for the $\mathcal{\mathcal{C} \mskip -4mu \mathcal{E}}$ machine.

There has also been recent interest in \emph{heapless} abstract
machines for lazy evaluation. Danvy et al. ~\cite{danvy2012inter} and
Garcia et al. ~\cite{garcia2009lazy} independently derived similar
machines from the call-by-need lambda calculus
~\cite{ariola1995call}. These are interesting approaches, but it is not yet
clear how these machines could be implemented efficiently.
