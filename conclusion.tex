\section{Conclusion} \label{sec:conc}

Lazy evaluation has long suffered from high overhead of delayed computations. By
carefully packaging these delayed computations, existing implementations pay an
up-front cost to ensure efficient variable lookup if a delayed computation is
entered. In this paper we have presented a novel approach that delays this cost,
so that if a delayed computation isn't entered, it isn't as costly. 

We would like the reader to take a few key points from this paper. First, a
shared environment, explicitly represented as a cactus stack, is a natural
way to share the results of computation as required by lazy evaluation. Second,
this approach is in a sense \emph{lazier} about lazy evaluation in that it
avoids some unnecessary packaging work. Third, this approach can be formalized
in both big-step and small-step semantics. Fourth, the abstract machine can be
implemented as a compiler in a straightforward way, yielding performance
comparable to existing implementations. 
