\section{Implementation} \label{sec:impl}

This section describes how the $\mathcal{CE}$ machine can be mapped directly to
x64 insructions. Specifically, we re-define the three instructions given by
the TIM~\cite{TIM}: \texttt{TAKE}, \texttt{ENTER}, and \texttt{PUSH}, and
implement them with x64 assembly. We also describe several design decisions, as
well as some optimizations. All implementation and benchmark code is available
at \texttt{http://cs.unm.edu/\textasciitilde stelleg/cem-tfp2016.tar.gz}.

Each closure is represented as a code pointer, environment pointer tuple. The
Context is implemented as a stack, with updates represented as a null pointer,
environment pointer tuple to differentiate them from closure arguments. The
Heap, or cactus environment, is implemented as a heap of environments. 

\subsection{Compilation}
The three instructions are given below, with descriptions of their behavior. 

\begin{itemize}
\item \texttt{TAKE}: Pops a context item off the stack. If the item is an
update $u:=$, the instruction updates the location $u$ with the current closure.
If it is an argument $c$, the instruction binds the closure $c$ to the fresh
location in the cactus environment.
\item \texttt{ENTER i}: Enters the closure defined by variable index \texttt{i},
the current environment pointer, and the current cactus environment.  \item
\texttt{PUSH m}: Pushes the code location \texttt{m} along with the
current environment pointer. 
\end{itemize}

These operations correspond directly to lambda terms, and to the transition
functions of the $\mathcal{CE}$ machine. The mapping from lambda terms can be
seen in Figure~\ref{fig:cemcompile}, which defines the compiler. Unlike the TIM,
our version of \texttt{TAKE} doesn't have an arity; we compile a sequence of
lambdas as a sequence of \texttt{TAKE} instructions. While have not compared
performance of the two approaches directly, we suspect that a \texttt{n} inlined
\texttt{TAKE} instructions should be roughly as fast as a \texttt{TAKE n}
instruction.  Similarly, the \texttt{ENTER i} instruction can be implemented
either as a loop or unrolled, depending on \texttt{i}, and more performance
comparisons are needed to determine the trade off between code size and speed.

\begin{figure}
\begin{align*} C[t \; t'] &= \texttt{PUSH}\;label_{C[t']}:C[t] \concat C[t'] \\
C[\lambda.t] &= \texttt{TAKE}:C[t] \\
C[i] &= \texttt{ENTER} \; i
\end{align*}
\caption{$\mathcal{CE}$ machine compilation scheme. $C$ compiles a sequence of
instructions from a term. The $label$ represents a code label: each instruction
is given a unique label. The $:$ operator denotes prepending an item to a
sequence and $\concat$ denotes concatenating two sequences.}
\label{fig:cemcompile}
\end{figure}

We compile to x64 assembly. Each of the three instructions is mapped onto
x64 instructions with a macro. The \texttt{PUSH} instruction is particularly
simple, consisting of only two x64 instructions (two \texttt{push}es, one for
the code pointer and one for the environment pointer). This is actually an
important point: \emph{thunk creation is only two hardware instructions,
regardless of environment size}.  

\subsection{Machine Literals and Primitive Operations}

Following Sestoft~\cite{sestoft}, we extend the $\mathcal{CE}$ machine to
include machine literals and primitive operations. Figure~\ref{fig:extsyntax}
shows the parts of syntax and semantics that are new or modified. 

\begin{figure}
\textbf{Syntax}
\begin{align*}
\tag{Term}    t &::= i \; | \; \lambda t \; | \; t \; t | \; m \\
\tag{Machine} m &::= n \; | \; op \\
\tag{Integer} n &\in \mathbb{I} \\
\tag{PrimOp} op &::= + \; | \; - \; | \; * \; | \; \; / \;\; | \; = \; | \; > \; | \; < \\
\tag{Value} v &::= \lambda t[l] \; | \; m[l] \\
\end{align*}
\textbf{Integer and Primop Semantics}
\begin{align*}
\tag{Int}
\langle n[l], \sigma \; c, \mu, k \rangle
  &\rightarrow_{\mathcal{CE}}
\langle c, \sigma \; n[l], \mu, k \rangle \\
\tag{Op} 
\langle op[l], \sigma \; n' \; n, \mu, k \rangle
  &\rightarrow_{\mathcal{CE}}
\langle op(n',n)[l], \sigma, \mu, k \rangle
\end{align*}
\caption{Extensions to the syntax and semantics of the $\mathcal{CE}$ machine.}
\label{fig:extsyntax}
\end{figure}

\subsection{Omitted Extensions}

Our implementation omits a few other standard extensions. Here we address some
of these omissions.

Data types are a common extension that we omit ~\cite{jonesstg,boquist1997grin}.
We take the approach of Sestoft ~\cite{sestoft} that these can be efficiently
implemented with pure lambda terms. For example, consider a list data type (in
Haskell syntax): \texttt{data List a = Cons a (List a) | Nil}. This can be
represented in pure lambda terms with $Cons = \lambda h.\lambda t.\lambda
c.\lambda n.c \; h \; t$ and $Nil = \lambda c.\lambda n.n$. 

Let bindings are another term commonly included in functional language
compilers, even in the internal representation ~\cite{boquist1997grin,jonesstg}.
Non-recursive let is syntactic sugar for a lambda binding and application, and
we treat it as such. This approach helps ensure that we can of
compile arbitrary lambda terms, while some approaches require pre-processing
~\cite{sestoft,TIM}.

Recursive let bindings are a third omission. Here we follow Rozas
~\cite{rozas1992taming}: If it can be represented in pure lambda terms, it should
be. Thus, we implement recursion using the standard Y combinator. In the case of
mutual recursion, we use the Y combinator in conjunction with a tuple of the
mutually recursive functions. Without the appropriate
optimizations ~\cite{rozas1992taming}, this approach has high overhead, as we
discuss in Section~\ref{sec:res}.

Our philosophy regarding extensions is: when an extension can be represented
with pure lambda calculus, then optimized to achieve equal performance to
implementations that represent them explicitly, then do so. This approach should
allow more general optimizations.

\subsection{Optimizations}

The $\mathcal{CE}$ implementation described in the previous section is 
completely unoptimized. For example, no effort is expended to
discover global functions to avoid costly jumps to pointers in the heap
~\cite{jonesstg}. Indeed, every variable reference will look up the code pointer
in the shared environment and jump to it. There is also no implementation of 
control flow analysis as used by Rozas to optimize away the Y combinator.  Thus,
every recursive call exhibits the large overhead involved in re-calculating the
fixed point of the function.  

We do, however, implement two basic optimizations, primarily to reduce the load
on the heap:

\begin{itemize}
\item \texttt{POP}: A \texttt{TAKE} instruction can be converted to a \texttt{POP}
instruction that throws away the operand on the top of the stack if there are no
variables bound to the $\lambda$ term in question. For example, the function
$\lambda x.\lambda y.x$ can be implemented with \texttt{TAKE}, \texttt{POP},
\texttt{ENTER 0}.  
\item \texttt{ENTERVAL}: An \texttt{ENTER} instruction, when entering a
closure that is already a value, should not push an update marker onto the
stack. This shortcut prevents unnecessary writes to the stack and heap
~\cite{jonesstg,lkm,sestoft}.  
\end{itemize}

\subsection{Garbage Collection}

We have implemented a simple mark and sweep garbage collector with the property
that it does not require two spaces, due to constant sized closures in the
heap, allowing a linked list representation for the free cells. Indeed,
while the abstract machine from Section~\ref{sec:mach} increments a free heap
counter, the actual implementation uses the next free cell in the linked list.

Because the focus of this paper is not on the performance of garbage collection,
we ensure the benchmarks in Section~\ref{sec:eval} are not dominated by GC time.

